\usepackage[utf8]{inputenc}
\usepackage[a4paper]{geometry}
\usepackage{lmodern}
\usepackage{microtype}

% Font stuff:
\usepackage{mathpazo} % Use the Palatino font.
\linespread{1.05}
\usepackage{tgcursor}

\usepackage{xargs}
\usepackage[pdftex, dvipsnames, usenames]{xcolor}
\usepackage[colorinlistoftodos, prependcaption, textsize=tiny]{todonotes}
\usepackage{upquote}
\usepackage{ellipsis}
\usepackage{booktabs}
\usepackage{parskip}
\usepackage{graphicx} % https://en.wikibooks.org/wiki/LaTeX/Importing_Graphics
\usepackage{dirtree}  % Required for directory trees.
\usepackage{listings} % Required for inserting code snippets.

\usepackage{enumitem} % http://tex.stackexchange.com/questions/10684/vertical-space-in-lists
\setlist[enumerate]{noitemsep}
\setlist[itemize]{noitemsep}
\setlist[description]{noitemsep}
% Use dash for items in lists:
\renewcommand{\labelitemi}{$-$}
\renewcommand{\labelitemii}{$-$}
\renewcommand{\labelitemiii}{$-$}
\renewcommand{\labelitemiv}{$-$}

\PassOptionsToPackage{hyphens}{url}
\usepackage[colorlinks=false, pdfborder={0 0 0}, unicode=true]{hyperref}
\urlstyle{same} % don't use monospace font for urls

% http://tex.stackexchange.com/questions/32208/footnote-runs-onto-second-page
\interfootnotelinepenalty=10000

% Own syntax highlighting:
% http://www.latextemplates.com/template/code-snippet
% http://www.stud.math.ntnu.no/kurs/latexdocs/listings.pdf
\lstset{
    basicstyle=\ttfamily, % The default font size and style of the code.
    breakatwhitespace=true, % If true, only allows line breaks at white space.
    breaklines=true, % Automatic line breaking (prevents code from protruding outside the box).
    tabsize=2, % Number of spaces per tab in the code
    captionpos=b, % Sets the caption position: b for bottom; t for top
    numbersep=10pt, % Distance of line numbers from the code box.
    numberstyle=\tiny\color{Gray}, % Style used for line numbers.
    firstnumber=1, % Line numbers begin at line 1.
    stepnumber=1, % The step distance between line numbers, i.e. how often will lines be    numbered.
}
\lstdefinelanguage{CayThe}{
    %escapeinside={\%}, % This allows you to escape to LaTeX using the character in the bracket
    keywordstyle=\textbf, 
    morekeywords={group, artifact, version, namespace, import, export, public, package, method},
    sensitive=true,
    numbers=none, % Location of line numbers, can take the values of: none, left, right
    showstringspaces=false, % Don't put marks in string spaces
    showtabs=false, % Display tabs in the code as lines.
    morestring=[b]",
    %stringstyle=\color{Purple}, % Strings are purple.
    morecomment=[l]{//}, 
    morecomment=[s]{/*}{*/},
}
\lstdefinelanguage{JavaScript}{
    keywordstyle=\textbf, 
    morekeywords={var, null, undefined, if, else, function},
    sensitive=true,
    numbers=none,
    showstringspaces=false,
    showtabs=false, 
    morestring=[b]",
    morecomment=[l]{//}, 
    morecomment=[s]{/*}{*/},
}

\lstdefinelanguage{FSharp}{
    keywordstyle=\textbf, 
    morekeywords={let, if, then, else, match, with, true, false},
    sensitive=true,
    numbers=none,
    showstringspaces=false,
    showtabs=false, 
}