\RequirePackage[l2tabu, orthodox]{nag}
\documentclass[11pt,a4paper]{report}

\usepackage[utf8]{inputenc}
\usepackage[a4paper]{geometry}
\usepackage{lmodern}
\usepackage{microtype}
\usepackage{xargs}
\usepackage[pdftex, dvipsnames]{xcolor}
\usepackage[colorinlistoftodos, prependcaption, textsize=tiny]{todonotes}
\usepackage{upquote}
\usepackage{ellipsis}
\usepackage{booktabs}
\usepackage{graphicx} % https://en.wikibooks.org/wiki/LaTeX/Importing_Graphics
\usepackage{dirtree}
\usepackage{enumitem} % http://tex.stackexchange.com/questions/10684/vertical-space-in-lists
\setlist{itemsep=.1em}
\PassOptionsToPackage{hyphens}{url}
\usepackage[colorlinks=false, pdfborder={0 0 0}, unicode=true]{hyperref}
\urlstyle{same} % don't use monospace font for urls

\renewcommand{\labelitemi}{$-$}
\renewcommand{\labelitemii}{$-$}
\renewcommand{\labelitemiii}{$-$}
\renewcommand{\labelitemiv}{$-$}

\title{%
    Cay-The Language Specification \\
    \large --- \\
    Version @pom.version@}
\author{Sven Strittmatter}
\date{\today}

\begin{document}

\maketitle

\begin{abstract}
    This document describes the language specification of the Cay-The language. This not only contains the syntax, but also the build tool chain, runtime environment, dependency management and standard library.
\end{abstract}

\tableofcontents

\chapter{Introduction}

This document describes in detail the whole specification of the Cay-The language. This does not only includes the syntax or the runtime environment, but also the whole module structure, dependency management and build and test tool chain.

\section{Conventions in this Document}

\subsection{Key Words to Indicate Requirement Levels}

The key words \textit{must}, \textit{must not}, \textit{required}, \textit{shall}, \textit{shall not}, \textit{should}, \textit{should not}, \textit{recommended}, \textit{may}, and \textit{optional} in this document are to be interpreted as described in RFC 2119\cite{rfc2119}.

\subsection{Syntax Definitions}

Syntax in any form must be be specified by grammars. Grammars are written in the ANTLR syntax\cite{antlr-docu}, e.g. (took from ANTLR examples\cite{antlr-grammars}):

\begin{verbatim}
    sentence    : subject predicate ;
    subject     : article? noun ;
    predicate   : verb adverb? ;
    article     : 'the' | 'a' | 'an' ;
    noun        : 'cat' | 'dog' | 'Mike' | 'Ireland' ;
    verb        : 'is' | 'are' | 'runs' ;
    adverb      : 'quickly' | 'slowly' ;
\end{verbatim}

This syntax is similar to \textit{Extended Backus–Naur form}\cite{ebnf-wiki}.

\chapter{General}

This section describes the top view: Modules and artifacts. It does not define a lot of syntax. Only the basic needed to define modules.

\section{Modules}

The top level artifact is a module which is simply a directory with a file called \texttt{Manifest.mf}.

\subsection{Directory Structure of a Module}

As mentioned the module is simply a directory with a manifest file in it. All types (will be described later) are also in this directory. Subdirectories build the package structure. The name of the directory is not the ``module name''. The module indeed does not have simply a name, rather than a so called \textit{coordinate}. This coordinate is a unique combination from the manifest file described later on. An example module directory may look like:

\dirtree{%
.1 module-dir.
.2 Manifest.mf.
.2 FileOne.ct.
.2 packgeone.
.3 FileTwo.ct.
.3 FileThree.ct.
}

\subsection{The Manifest}

The manifest file must have the name \texttt{Manifest.mf} in the top level directory of the module. Also it must contain some essential key information to describe the module:

\begin{itemize}
    \item \texttt{group}: The group name is a name to put some artifacts together. The naming scheme is a reversed domain name: a so called \textit{full qualified name}. For example you have the \textit{Fancy Lib} project which is hosted under \textit{fancy-lib.org} then your group should look like \textit{org.fancy-lib}. If you have multiple projects then it may be reasonable to make a group for each of them. E.g.\ your organisation (\textit{fancy.org}) has two projects: \textit{fancy-lib} and \textit{fancy-framework} then your groups may be \textit{org.fancy.fancy-lib} and \textit{org.fancy.fancy-framework}.
    \item \texttt{artifact}: The artifact is a name of a single module belonging to a group. There exists at least one artifact with a unique artifact name and a unique group name. But there also may exist more than one module with different artifact names but the same group name. So for example the above mentioned \textit{fancy lib} has a \textit{core} and a \textit{plugin} module, then you would have two directories with each a manifest file: One with artifact name \textit{core} and one with the artifact name \textit{plugin}, but both with the group name \textit{org.fancy-lib}.
    \item \texttt{version}: The version string follows the rules of Semantic Versioning\cite{semver}.
    \item \texttt{namespace}: The namespace is also a full qualified name which is prefixed to everything declared in the module. So if you have the \textit{fancy lib} example above you ma choose the namespace \textit{org.fancy-lib.core}. A good starting point to find a proper namespace is to use a combination of group and artifact name. Then a type \textit{T} declared and exported by the module must be imported by other modules via \texttt{import org.fancy-lib.core.T}. But the file declaring the type \textit{T} is placed in the top level directory of the module. So you do not need to create a ``package structure'' for your namespace with directories. 
\end{itemize}

All these manifest \textit{directives} must be present once and the values must not be blank. The grammar rule for a \textit{full qualified name} is is: 

\begin{verbatim}
    fullQualifiedName : IDENTIFIER ('.' IDENTIFIER)* ;    
    IDENTIFIER : LETTER (CHARACTER | '-')* ;
\end{verbatim}

\noindent
An example manifest may look like:

\begin{verbatim}
    group       de.weltraumschaf
    artifact    example
    version     1.0.0
    namespace   de.weltraumschaf.example
\end{verbatim}

\subsection{Coordinate}

Every module and its final assembled artifact has a unique coordinate to identify them. The coordinate is a combination of \textit{group}, \textit{artifact}, and \textit{version}. So for the example above the coordinate would be \texttt{de.weltraumschaf:example:1.0.0}. The parts of the coordinate are separated by a colon. So the grammar rule for a coordinate is\footnote{IDENTIFIER is defined already above for the full qualified name rule.}:

\begin{verbatim}
    coordinate
        :
            group=fullQualifiedName ':'
            artifact=fullQualifiedName ':'
            version
        ;
    version
        : 
            major=NUMBER '.' minor=NUMBER '.' patch=NUMBER
            ('-' identifiers=IDENTIFIER)?
        ;
    NUMBER : DIGIT+ ;
    DIGIT  : [0-9] ;
\end{verbatim}

\subsection{Imports}

A manifest also may contain \textit{import} directives to make other artifacts and their exported types available in the module. The import is simply done by the coordinate of the desired module. E.g.\ the above manifest now importing two other modules:

\begin{verbatim}
    group       de.weltraumschaf
    artifact    example
    version     1.0.0
    namespace   de.weltraumschaf.example

    import      de.weltraumschaf:core:1.2.3
    import      de.weltraumschaf:test:2.0.0
\end{verbatim}

Coordinates must not be duplicated in the imports.

\subsection{Submodules}

Not specified yet.

\subsection{Exported Imports}

Not specified yet.

\chapter{Ideas}

This chapter contains ideas for the work in progress. This is not part of the official specification!

\begin{itemize}
    \item \texttt{unless foo} statt \texttt{if !foo}
    \item type (int, float) overflows generate exceptions
    \item compiler checks for right versioning in manifest
    \item compiler does not allow shot names of identifiers
    \item \texttt{finally} to execute something always at least in functions
\end{itemize}

\section{Source}

A source file \texttt{MyType.ct} may look like:

\begin{verbatim}
var integer foo = 23
var integer bar // is 0 by default
var integer baz = foo + bar

const snafu = "Hello !"

function doIt(Integer i) {
    ...
}

function Integer doWhat(Integer i, Integer j) {
    return i + j
    
    finally {
        // Something which is done before function returns, e.g. clean resources.
    }
}

function Integer multiFinallies() {
    File f = new File("/foo");
    f.open();

    finally {
        f.close()
    }

    File d = new File("/tmp");
    d.open()

    finally {
        d.close()
    }

    return 42
}

var result = doWhat(foo, 42)	
\end{verbatim}

\section{Specification}

\section{Tooling}

A perfect language need tools. Some of them are obvious:

\begin{itemize}
    \item Interpreter: An interpreter is useful in the first steps of language development, but also useful later to provide endusers a REPL (read eval print loop) tool for playing around with the language without ramp up a whole project and build infrastructure.
    \item Compiler: Of course there must be a compiler to create byte code for a virtual machine.
    \item Virtual Machine: For interpreting byte code a virtual machine is necessary. This is implemented as register based virtual machine.
\end{itemize}

The not so obvious tools are a level above just compiling source code.

\begin{itemize}
    \item Formatter: Most languages say not much about how to format it. Therefore almost mostly formatting is a very opinion based thing. This leads to endless discussions in projects involving more than one programmer. Thats the reason why this language provides a formatter with a default which is highly recommended to use.
    \item Tester: There is tool which executes tests automatically. Also there is a built in API for unit testing.
    \item Builder: This is a tool which compiles and links a complete module. It also runs the Tester to verify the tests. Also it manages dependencies of the built module.
\end{itemize}

\section{Module Packaging and Visibility}

A module is a directory containing all source code. It contains one and only one manifest file which declares basic attributes for the module:

\begin{itemize}
    \item \texttt{group}: This declares the group name of the module.
    \item \texttt{artifact}: This declares the artifact name of the module.
    \item \texttt{version}: This declares the version of the module.
    \item \texttt{namespace}: This declares the namespace base of all types in the module.
\end{itemize}

Also the manifest declares the module dependencies. Modules are referenced by an import and a module coordinate.

An example of a manifest file \texttt{moduledir/Manifest.mf}:
\begin{verbatim}
    group       de.weltraumschaf
    artifact    example
    version     1.0.0
    namespace   de.weltraumschaf.example

    import      org.caythe:core:1.0.0
    import      org.caythe:testing:1.0.0
\end{verbatim}

All types declared in the root of the module directory are referenced by the package name given by the namespace declaration. All sub directories are packages. For example a class \texttt{Foo} in the file \texttt{moduledir/Foo.ct} will be referenced by \texttt{use de.weltraumschaf.example.Foo}. A type is always declared in a single file and a single file can only declare one type. So the File \texttt{Foo.ct} declares the type \texttt{Foo}. In consequence it is not necessary to type the name of the type again in side the file and renaming the file result in renaming the type.

From outside modules only types with explicit \texttt{export} are visible. So that other modules which import \texttt{de.weltraumschaf:example:1.0.0} may see the type \texttt{Foo} it must be declared exported.

Also each method which should be accessible from outside must be declared exported. Methods or types declared \texttt{public} are only visible everywhere inside the module. Everything declared \texttt{package} is only visible in its directory and all sub directories. So a type or method in the root module directory declaring \texttt{package} has the same effect as declaring it \texttt{public}. By default methods and types are private.

\begin{verbatim}
    export // Exports the type
    public // Marks the type public

    export method callableFromOtherModules() { ... }
    public method callableFromWholeModule() { ... }
    package method callableFromSameDirAndSubDirs() { ... }
    // private is the default so there is no keyword for that.
    method callableOnlyFromSelf() { ... }
\end{verbatim}

\todo{Describe the followed items.}

\begin{itemize}
    \item Types
    \begin{itemize}
        \item integer: 0 by default
        \item float: 0.0 by default
        \item boolean: false by default
        \item string: empty by default
    \end{itemize} 
    \item Scopes
    \begin{itemize}
        \item Lexical Scopes
        \item Variables are only writable in current scope. What about members of classes?
    \end{itemize}
\end{itemize}

\clearpage

\begin{thebibliography}{100}

\bibitem{antlr-docu}
    ANTLR 4,
    \textit{ANTLR 4 Documentation},
    \url{https://github.com/antlr/antlr4/blob/master/doc/index.md}

\bibitem{antlr-grammars}
    ANTLR 4,
    \textit{Grammars written for ANTLR v4},
    \url{https://github.com/antlr/grammars-v4}

\bibitem{semver}
    Semantic Versioning,
    \textit{Semantic Versioning 2.0.0},
    \url{http://semver.org}

\bibitem{ebnf-wiki}
    Wikipedia,
    \textit{Extended Backus–Naur form},
    \url{https://en.wikipedia.org/wiki/Extended_Backus\%E2\%80\%93Naur_form}

\bibitem{rfc2119}
    S. Bradner,
    \textit{Key words for use in RFCs to Indicate Requirement Levels},
    \url{https://tools.ietf.org/html/rfc2119}

\end{thebibliography}

\end{document}
