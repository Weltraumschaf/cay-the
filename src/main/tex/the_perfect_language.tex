\documentclass[a4paper,12pt]{article}

\usepackage{hyperref}

\title{The Perfect Programming Language}
\author{Sven Strittmatter}
\date{\today}

\begin{document}
\maketitle
\abstract{This paper describes the perfect programming language. We will look at drawback of current programming languages. Then try to describe a programming language  without these drawbacks.}
\newpage

\tableofcontents
\newpage

\section{Motivation}

What is my motivation behind this: Creating a new perfect programming language? As of time of writing this I have over a decade of experience in programming stuff. I also have experience in various languages: C++, PHP, JavaScript, Ruby, Perl, Java. Also I saw a lot of Languages. After some disappointment about the languages I used to looked around what others do: Go, Python, Erlang, Prolog, Lisp, ML, OCaml, Scala etc.

More and more I saw different languages and their concepts I recognised that most of them have some drawbacks and I started to think about a programming language without any of these. I'm not shur if it is possible to make such a language, but I think it is worth to think about it.

\subsection{Drawbacks and Fails}

In this section I'll collect all the drawbacks and fails already existing languages have. These are not drawbacks only because I say. All of them have some common sense in the in the community of software craftmenship, clean code developers or others working hard on better software and design. So the points mentioned here are quite common sense for all well exercised developers. In a later section I will describe "disappointing" things which are merely based on my opinion.

\subsubsection{Null - The Billion Dollar Fail}

What the inventor Tony Hoare\cite{hoare-wiki} said about null\cite{hoeare-null}:

\begin{quotation}
I call it my billion-dollar mistake. It was the invention of the null reference in 1965. At that time, I was designing the first comprehensive type system for references in an object oriented language (ALGOL W). My goal was to ensure that all use of references should be absolutely safe, with checking performed automatically by the compiler. But I couldn't resist the temptation to put in a null reference, simply because it was so easy to implement. This has led to innumerable errors, vulnerabilities, and system crashes, which have probably caused a billion dollars of pain and damage in the last forty years.	
\end{quotation} 

TODO Cite some sites why null is bad. Null is one of the most reasons for bugs.

\subsubsection{Checked Exceptions}

TODO Words from Andre Heilsberg (C\#)

\subsubsection{Exceptions at All}

TODO They are like goto (Javaslang)

\subsection{Disappointing}

In this section I describe things which are very disappointing, e.g. typing boiler plate code, parens, semicolons and such. In contrast to the section above these points are merely opinion based. But I try to show why we should think about it.

\subsubsection{Annoying Braces}

TODO: Braces for conditions: if (condition) vs. if condition

\subsubsection{Unnecessary Semicolon}

TODO

\section{What Ideas Are Out There?}

In this section ideas and concepts of already existing languages will be described.

\subsection{Kotlin}

From a blog post from Peter Sommerhoff\cite{kotlin-sommerhoff}:

\subsubsection{Data Classes}

Simple POJOs are declared simpler. Instead of writing boiler plate like:

\begin{verbatim}
class Book {
    private String title;
    private Author author;

    public String getTitle() {
        return title;
    }
    
    public void setTitle(String title) {
        this.title = title;
    }

    public Author getAuthor() {
        return author;
    }
    
    public void setAuthor(Author author) {
        this.author = author;
    }
}
\end{verbatim}

You can write:

\begin{verbatim}
data class Book(var title: String, var author: Author) {
    // ...
}	
\end{verbatim}

\subsubsection{Smart Casts}

Instead of:

\begin{verbatim}
if (node instanceof Leaf) {
    return ((Leaf) node).symbol;
}	
\end{verbatim}

less verbose:

\begin{verbatim}
if (node is Leaf) {
    return node.symbol;
}	
\end{verbatim}

\subsubsection{Type Inference}
TODO

\subsubsection{Functional Programming}
TODO

\subsubsection{Default Arguments}
TODO

\subsubsection{Named Arguments}
TODO

\subsubsection{Final Classes}

\begin{quotation}
Next, Kotlin also supports the principle to either design for inheritance or prohibit it — because in Kotlin, you have to explicitly declare a class as open in order to inherit from it. That way, you have to remember to allow inheritance instead of having to remember to disallow it.\cite{kotlin-sommerhoff}
\end{quotation}

\section{Ideas}

\begin{itemize}
	\item \texttt{unless foo} statt \texttt{if !foo}
	\item type (int, float) overflows generate exceptions
	\item compiler checks for right versioning in manifest
	\item compiler does not allow shot names of identifiers
	\item \texttt{finally} to execute something always at least in functions
\end{itemize}

\subsection{Source}

A source file \texttt{MyModule.ct} may look like:

\begin{verbatim}
var integer foo = 23
var integer bar // is 0 by default
var integer baz = foo + bar

const snafu = "Hello !"

function doIt(Integer i) {
    ...
}

function Integer doWhat(Integer i, Integer j) {
    return i + j
    
    finally {
    	// Something which is done before function returns, e.g. clean resources.
    }
}

function Integer multiFinallies() {
    File f = new File("/foo");
    f.open();
    
    finally {
        f.close()
    }
    
    File d = new File("/tmp");
    d.open()
    
    finally {
    d.close()
    }
    
    return 42
}

var result = doWhat(foo, 42)	
\end{verbatim}

\subsection{Manifest and Versioning}

A module descriptor \texttt{MyModule.mf} may look like:

\begin{verbatim}
version 1.0.0

export TypeName
export TypeName
export TypeName

Integer main(List<String> args) {
  return 0
}
\end{verbatim}

\section{Specification}

\subsection{Tooling}

A lot of languages lack of appropriate tooling and it is necessary to build them first you get productive. A perfect language must come with all necessary tools. Some of them are obvious:

\begin{itemize}
	\item Interpreter: An interpreter is useful in the first steps of language development, but also useful later to provide endusers a REPL (read eval print loop) tool for playing around with the language without ramp up a whole project and build infrastructure.
	\item Compiler: Of course there must be a compiler to create byte code for a virtual machine.
	\item Virtual Machine: For interpreting byte code a virtual machine is necessary. This is implemented as register based VM.
\end{itemize}

The not so obvious tools are a level above just compiling source code.

\begin{itemize}
	\item Formatter: Most languages say not much about how to format it. Therefore almost mostly formatting is a very opinion based thing. This leads to endless discussions in projects involving more than one programmer. Thats the reason why this language provides a formatter with a default which is highly recomented to use.
	\item Tester: There is tool which executes tests automatically. Also there is a built in API for unit testing.
	\item Builder: This is a tool which compiles and links a complete module. It also runs the tester to verify the tests. Also it manages dependneeies of the built module.
\end{itemize}

\subsection{Module Packaging and Visibility}

A module is a directory containing all source code. It contains one and only one manifest file which declares basic attributes for the module:

\begin{itemize}
	\item \verb|group|: This declares the group name of the module.
 	\item \verb|artifct|: This declares the artifact name of the module.
 	\item \verb|version|: This declares the version of the module.
 	\item \verb|namespace|: This declares the namespace base of all types in the module.
\end{itemize}

Also the manifest declares the module dependencies. Modules are referenced by an import and a module coordinate.

An example of a manifest file \verb|moduledir/Manifest.mf|:
\begin{verbatim}
	group       de.weltraumschaf
	artifact	    example
	version     1.0.0
	namespace   de.weltraumschaf.example
	import      org.caythe:core:1.0.0
	import      org.caythe:testing:1.0.0
\end{verbatim}

All types declared in the root of the module directory are referenced by the package name given by the namespace declaration. All usb directories are packages. For example a class \verb|Foo| in the file \verb|moduledir/Foo.ct| will be referenced by \verb|use de.weltraumschaf.example.Foo|.

From outside modules only types with explicit \verb|export| are visible. So that other modules which import \verb|de.weltraumschaf:example:1.0.0| may see the type \verb|Foo| it must be declared exported:
\begin{verbatim}
	export class Foo {
	
	}
\end{verbatim}

Also each method which should be accessible from outside must be declared exported. Methods or types declared \verb|public| are only visible everywhere inside the module. Everything declared \verb|package| is only visible in its directory and all sub directories. So a type or method in the root module directory declaring \verb|package| has the same effect as declaring it \verb|public|. By default methods and types are private.

\begin{verbatim}
	export class Foo {
	    export method callableFromOtherModules() { ... }
	    public method callableFromWholeModule() { ... }
	    package method callableFromSameDirAndSubDirs() { ... }
	    // private is the default so there is no keyword for that.
	    method callableOnlyFromSelf() { ... }
	}
\end{verbatim}

TODO:

\begin{itemize}
	\item Types
  	\begin{itemize}
  		\item integer: 0 by default
  		\item float: 0.0 by default
  		\item boolean: false by default
  		\item string: empty by default
  	\end{itemize} 
  	\item Scopes
  	\begin{itemize}
  		\item Lexical Scopes
  		\item Variables are only writable in current scope. What about members of classes?
  	\end{itemize}
\end{itemize}

\newpage
\begin{thebibliography}{56}
\bibitem{hoare-wiki}
	Wikipedia,
	\emph{Tony Hoare},
	\url{https://en.wikipedia.org/wiki/Tony_Hoare}
	
\bibitem{hoeare-null}
	Tony Hoare,
	\emph{Speaking at a conference in 2009},
	\url{http://www.infoq.com/presentations/Null-References-The-Billion-Dollar-Mistake-Tony-Hoare}
	
\bibitem{kotlin-sommerhoff}
 	Peter Sommerhoff,
 	\emph{Kotlin for Java Developers: 10 Features You Will Love About Kotlin}
 	\url{https://www.javacodegeeks.com/2016/05/kotlin-java-developers-10-features-will-love-kotlin.html}
\end{thebibliography}
\end{document}