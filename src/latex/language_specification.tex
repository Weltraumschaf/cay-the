\documentclass[a4paper,12pt]{article}

\usepackage[utf8]{inputenc}
\usepackage{hyperref}
\usepackage{xargs}
\usepackage[pdftex,dvipsnames]{xcolor}
\usepackage[colorinlistoftodos,prependcaption,textsize=tiny]{todonotes}

\newcommandx{\myfixme}[2][1=]{\todo[linecolor=red,backgroundcolor=red!25,bordercolor=red,#1]{#2}}
\newcommandx{\mytodo}[2][1=]{\todo[linecolor=blue,backgroundcolor=blue!25,bordercolor=blue,#1]{#2}}
\newcommandx{\myxxx}[2][1=]{\todo[linecolor=OliveGreen,backgroundcolor=OliveGreen!25,bordercolor=OliveGreen,#1]{#2}}

\title{Language Specification}
\author{Sven Strittmatter}
\date{\today}

\begin{document}

\maketitle
\begin{abstract}
    This document describes the language specification of the Cay-The language.
\end{abstract}
\clearpage

\tableofcontents
\clearpage

\section{General}

TODO

\subsection{Modules}

The top level artifact is a module which is simply a directory.

TODO 

\subsubsection{Directory Structure of a Module}

\begin{verbatim}
    modulename/
        +- Manifest.mf
        +- FileOne.ct
        +- packgeone
            +- fileTwo.ct
            +- FileThree.ct
\end{verbatim}

\subsubsection{The Manifest}

TODO 

\begin{verbatim}
    group       de.weltraumschaf
    artifact    example
    version     1.0.0
    namespace   de.weltraumschaf.example
    import      org.caythe:core:1.0.0
    import      org.caythe:testing:1.0.0
\end{verbatim}
\clearpage


\begin{thebibliography}{56}
    
\bibitem{lorem}
    Lorem
    \textit{Lorem ipsum},
    \url{https://en.wikipedia.org/}
        
\end{thebibliography}

\end{document}